\documentclass [11pt,a4paper,oneside]{article}

\usepackage[italian]{babel}
\usepackage[utf8]{inputenc}
\usepackage{indentfirst}
\usepackage{graphicx}
\usepackage{geometry}
\usepackage{verbatim}
\usepackage{color}
\usepackage{listings}

% Informations
\title{Odometro incrementale}
\author{Emanuele Aina (matr. 129548)}
\date{}

% Margins
\geometry{bindingoffset=0.5cm}

% Caption
\newcommand{\mc}[1]{{\small\ #1}}
\newcommand{\makecaption}[1]{\caption{\mc\ #1}}

% URL
\newcommand{\url}[2]{#1\footnote{#2}}

% References
\newcommand{\refemph}[1]{\emph{#1}}
\newcommand{\imgref}[1]{\refemph{\ref{img:#1}}}
\newcommand{\imgrefx}[1]{\refemph{figura}~\imgref{#1}}

% Images
\newcommand{\im}[3]{
    \begin{figure}[!htb]
        \begin{center}
            \includegraphics{img/#1}
            \makecaption{#2}
            \label{img:#3}
        \end{center}
    \end{figure}
}

% Collegamenti.
\definecolor{links}{rgb}{0.7,0.2,0}
\usepackage{ifpdf}
\ifpdf
    \usepackage{hyperref}
    \hypersetup{colorlinks=true,bookmarks=true,hypertexnames=false,urlcolor=links}
\else
    \newcommand{\href}[2]{#2}
\fi
\newcommand{\link}[1]{\href{#1}{#1}}

% Ignore the argument
\newcommand{\noop}[1]{}

% Foreign words
\newcommand{\foreign}[1]{\emph{#1}}

\newcommand{\component}[1]{\texttt{#1}}
\newcommand{\identifier}[1]{\texttt{#1}}

% Justify even if it has to leave a lot of blank space
\setlength{\emergencystretch}{3em}

\definecolor{comment}{rgb}{0,0,0.8}
\definecolor{keyword}{rgb}{0.9,0.5,0}
\definecolor{string}{rgb}{0.95,0,0}

\lstset{
    language=VHDL,
    basicstyle=\ttfamily,
    % this sets the letter spacing as it is with the verbatim package
    basewidth=0.5em,
    keywordstyle=\color{keyword},
    commentstyle=\color{comment},
    stringstyle=\color{string},
    showstringspaces=false,
}


\begin{document}

\maketitle

\begin{center}
\small{01XXX - Metodi e strumenti di coprogettazione di sistemi digitali }
\end{center}

\section{Specifiche di progetto}
Realizzare un odometro rotativo assoluto, da connettere ad un encoder
incrementale con ingresso di 0 (A, B, Z: segnali TTL). Deve misurare l'angolo
(bidirezionale) e il numero di giri, acquisibili via RS232, e fornire un
impulso su un'uscita TTL ogni K gradi (programmabile), solo quando l'angolo
sia compreso fra K1 e K2 gradi (entrambi programmabili). Deve essere
programmabile (via RS232) per impostare il coefficiente angolare dell'encoder
(gradi per impulso) e le costanti K, K1 e K2. Risoluzione 0.1 grado, angolo
massimo 1000 giri. Alimentazione 5VDC+/-5\%.

\section{Analisi dei casi d'uso}
\begin{center}
    \includegraphics[width=12cm]{uml/odometer_usecases.pdf}
    \label{usecases}
\end{center}

Gli attori rappresentano le due sorgenti di input dell'Odometro,
ovvero l'interfaccia seriale RS232 e l'output dell'Encoder incrementale.

\begin{itemize}
\item[Invio via seriale] RS232 invia all'Odometro dei dati tramite l'interfaccia
     seriale di quest'ultimo.
\item[Invio comando] I dati inviati da RS232 all'interfaccia seriale dell'Odometro
     rappresentano un comando che deve essere decodificato.
\item[Invio dati] A seguito di un comando che richiede dati aggiuntivi per essere
     eseguito, RS232 provvede a fornire i dati necessari all'Odometro.
\item[Ricezione via seriale] A seguito di un comando di lettura, RS232 attende di
     ricevere i dati richiesti all'Odometro.
\item[Lettura angolo] Il comando inviato all'Odometro richiede la lettura del
     valore corrente dell'angolo e questo viene ricevuto tramite l'interfaccia
     seriale.
\item[Lettura rivoluzioni] Il comando inviato all'Odometro richiede la lettura del
     valore corrente del contatore delle rivoluzioni e questo viene ricevuto
     tramite l'interfaccia seriale.
\item[Imposta coefficiente] Il comando inviato all'Odometro richiede l'impostazione
     del coefficiente angolare corrispondente a un segnale dell'Encoder e il valore
     viene trasmesso attraverso l'interfaccia seriale.
\item[Imposta K] Il comando inviato all'Odometro richiede l'impostazione del
     parametro K e il valore viene trasmesso attraverso l'interfaccia seriale.
\item[Imposta K1] Il comando inviato all'Odometro richiede l'impostazione del
     parametro K1 e il valore viene trasmesso attraverso l'interfaccia seriale.
\item[Imposta K2] Il comando inviato all'Odometro richiede l'impostazione del
     parametro K2 e il valore viene trasmesso attraverso l'interfaccia seriale.
\end{itemize}


\section{Implementazione}


\section{Ulteriori informazioni}
L'intero progetto è disponibile presso \link{http://techn.ocracy.org/odometer}, da cui
è inoltre possibile visionare tutte le revisioni realizzate durante lo sviluppo.
Il repository contiene inoltre i sorgenti di questo documento e il Makefile
per GNU~make usato per automatizzare il processo di compilazione e test.

\end{document}
